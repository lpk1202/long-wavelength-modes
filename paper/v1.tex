\documentclass[prd,amsmath,amssymb,floatfix,superscriptaddress,nofootinbib,twocolumn]{revtex4-1}

\def\be{\begin{equation}}
\def\ee{\end{equation}}
\def\bea{\begin{eqnarray}}
\def\eea{\end{eqnarray}}
\newcommand{\vs}{\nonumber\\}
\newcommand{\vk}{\vec{k}}
\newcommand{\ec}[1]{Eq.~(\ref{eq:#1})}
\newcommand{\eec}[2]{Eqs.~(\ref{eq:#1}) and (\ref{eq:#2})}
\newcommand{\Ec}[1]{(\ref{eq:#1})}
\newcommand{\eql}[1]{\label{eq:#1}}
\newcommand{\sfig}[2]{
\includegraphics[width=#2]{../plots/#1}
        }
\newcommand{\sfigr}[2]{
\includegraphics[angle=270,origin=c,width=#2]{#1}
        }
\newcommand{\sfigra}[2]{
\includegraphics[angle=90,origin=c,width=#2]{#1}
        }
\newcommand{\Sfig}[2]{
   \begin{figure}[thbp]
   \begin{center}
    \sfig{#1.pdf}{\columnwidth}
    \caption{{\small #2}}
    \label{fig:#1}
     \end{center}
   \end{figure}
}
\newcommand{\Spng}[2]{
   \begin{figure}[thbp]
   \begin{center}
    \sfig{#1.png}{\columnwidth}
    \caption{{\small #2}}
    \label{fig:#1}
     \end{center}
   \end{figure}
}

\usepackage[utf8]{inputenc}
\usepackage{graphicx}
\usepackage{amssymb}
\usepackage{simplewick}
\usepackage{amsmath}
\usepackage{bm}
\usepackage{color}
\usepackage{enumitem}
\usepackage[linktocpage=true]{hyperref} 
\hypersetup{
    colorlinks=true,       
    linkcolor=red,         
    citecolor=blue,        
    filecolor=magenta,      
    urlcolor=blue           
}
\usepackage[all]{hypcap} 
\definecolor{darkgreen}{cmyk}{0.85,0.1,1.00,0} 
\newcommand{\scott}[1]{{\color{darkgreen} #1}}
\newcommand{\peikai}[1]{{\color{blue} #1}}
\newcommand{\WH}[1]{{\color{red} #1}}
\newcommand{\wh}[1]{{\color{red} #1}}
\newcommand{\prvs}[1]{{\color{magenta} #1}}
\newcommand{\AL}[1]{{\color{magenta} AL: #1}}
\newcommand{\MR}[1]{{\color{blue} MR: #1}}
\newcommand{\nl}{\\ \indent}
\begin{document}
\title{Large Scale Modes Reconstruction with Short-Wavelength Modes}
\author{\large Peikai Li}
\author{\large Scott Dodelson}
\author{\large XXX}
\affiliation{Department of Physics, Carnegie Mellon University, Pittsburgh, Pennsylvania 15312, USA}

\date{\today}
\begin{abstract}

\end{abstract}
\maketitle
\section{Introduction}
\cite{Hu:2002mr}

\section{Standard Perturbation Theory}
In this section we review the SPT\footnote{SPT stands for standard perturbation theory.} approach of describing small scale nonlinearity. In the case of a perfect pressurelesee fluid, the nonrelativistic cosmological fluid equations are continuity, Euler and Poisson equations:
\bea
\frac{\partial \delta(\vec{x},\tau)}{\partial \tau} &+&\vec{\nabla}\times [(1+\delta(\vec{x},\tau))\vec{v}(\vec{x},\tau)] =0 \\
(\frac{\partial}{\partial \tau} +\vec{v}(\vec{x},\tau)&\times&\vec{\nabla})\vec{v}(\vec{x},\tau)=-\frac{da}{d\tau}\frac{\vec{v}(\vec{x},\tau)}{a}-\vec{\nabla}\Phi \eql{eulereq}\\
&&\nabla^2 \Phi = 4\pi G a^2 \bar{\rho}_{\rm m} \delta(\vec{x},\tau) 
\eea
these equations fully determine the time evolution of the local density contrast $\delta$ and the peculiar velocity field $\vec{v}=d\vec{x}/d\tau$. Taking the divergence of \ec{eulereq} we can eliminate the gravitational potential $\Phi$ in Fourier space and get:
\bea 
\frac{\partial{\delta}}{\partial\tau} + {\theta} &=& -\int \frac{d^{3}\vk_{1}}{(2\pi)^3} \int \frac{d^{3}\vk_{2}}{(2\pi)^3} (2\pi)^3 \delta_{\rm D}(\vk-\vk_{12}) \vs
&& \times\frac{\vk \cdot \vk_1}{k_1^2}{\theta}(\vk_1,\tau){\delta}(\vk_2,\tau) \eql{fouriercont}\\
\frac{\partial{\delta}}{\partial\tau}+\frac{da}{d\tau}\frac{{\theta}}{a}+\frac{6}{\tau^2}{\delta}&=&-\int \frac{d^{3}\vk_{1}}{(2\pi)^3} \int \frac{d^{3}\vk_{2}}{(2\pi)^3} (2\pi)^3 \delta_{\rm D}(\vk-\vk_{12})\vs
&&\times \frac{k^2(\vk_1 \cdot \vk_2)}{2k_1^2 k_2^2} {\theta}(\vk_1,\tau){\theta}(\vk_2,\tau) \eql{fouriereuler}
\eea
here $\vk_{12}=\vk_1 +\vk_2$ and more generally $\vk_{1\cdots n}=\vk_1 + \cdots +\vk_n$; $\theta\equiv \vec{\nabla}\cdot \vec{v}$ is the divergence of velocity field.\\
In Einstein-de Sitter space, linear growth function $D_{1}(a)=a$ and we can solve these equations perturbatively using expansion \cite:
\bea
{\delta}(\vk,\tau)&=&\sum_{n=1}^{\infty} {\delta}^{(n)}(\vk,\tau)=\sum_{n=1}^{\infty}a^{n}(\tau)\delta_{n}(\vk) \\ {\theta}(\vk,\tau)&=&\sum_{n=1}^{\infty}{\theta}^{(n)}(\vk,\tau)=-H(\tau)\sum_{n=1}^{\infty}a^{n+1}(\tau)\theta_{n}(\vk) 
\eea
\\where the superscript $(n)$ means the order of perturbation theory, and first order term ${\delta}^{(1)}$ corresponds to linear evolution. Linear power spectrum is given by this term via:
\be 
\langle {\delta}(\vk){\delta}(\vk') \rangle =(2\pi)^3 \delta_{\rm D}(\vk+\vk')P_{\rm lin}(k) \eql{lin}
\ee 
Substituting the two perturbative series into \ec{fouriercont} and \ec{fouriereuler} we get recursion relations for ${\delta}_{n}(\vk)$ and ${\theta}_{n}(\vk)$ \cite{cite} with solution:
\bea
{\delta}_{n}(\vk) &=& \int \frac{d^{3}\vk_{1}}{(2\pi)^3} \cdots \int \frac{d^{3}\vk_{n}}{(2\pi)^3}(2\pi)^3 \delta_{\rm D} (\vk-\vk_{1\cdots n}) \vs
&&\times F_{n}(\vk_1,\cdots,\vk_n){\delta}_1(\vk_{1}) \cdots {\delta}_1(\vk_{n}) \\
{\theta}_{n}(\vk) &=& \int \frac{d^{3}\vk_{1}}{(2\pi)^3} \cdots \int \frac{d^{3}\vk_{n}}{(2\pi)^3}(2\pi)^3 \delta_{\rm D} (\vk-\vk_{1\cdots n}) \vs
&&\times G_{n}(\vk_1,\cdots,\vk_n){\delta}_1(\vk_{1}) \cdots {\delta}_1(\vk_{n})
\eea
and recursion relations are encoded in kernels $F_n$ and $G_n$. Since we only consider up to second-order, the second-order symmetrized kernels are given by:
\bea 
F_{2}(\vk_1,\vk_2)=\frac{5}{7}+\frac{2}{7}\frac{(\vk_1\cdot \vk_2)^2}{k_1^2 k_2^2}+\frac{\vk_1\cdot \vk_2}{2k_1k_2}(\frac{k_1}{k_2}+\frac{k_2}{k_1}) \\
G_{2}(\vk_1,\vk_2)=\frac{3}{7}+\frac{4}{7}\frac{(\vk_1\cdot \vk_2)^2}{k_1^2 k_2^2}+\frac{\vk_1\cdot \vk_2}{2k_1k_2}(\frac{k_1}{k_2}+\frac{k_2}{k_1})
\eea 
This approach gives a nearly accurate description of $\Lambda$CDM universe \cite{123} with a slight generalization:
\bea
{\delta}(\vk,\tau) &=&\sum_{n=1}^{\infty} {\delta}^{(n)}(\vk,\tau)=\sum_{n=1}^{\infty}D_1^{n}(\tau)\delta_{n}(\vk) \\
{\theta}(\vk,\tau)&=&\sum_{n=1}^{\infty}{\theta}^{(n)}(\vk,\tau)\vs
&=&-\frac{d\,\ln D_1(\tau)}{d\tau}\sum_{n=1}^{\infty}D_1^{n}(\tau)\theta_{n}(\vk) 
\eea
Time-dependent second-order density contrast can be expressed as convolution of two linear density fields with kernel $F_2$:
\be 
{\delta}^{(2)}(\vk,\tau)=\int \frac{d^{3}\vk_{1}}{(2\pi)^3} F_2(\vk_1,\vk-\vk_1){\delta}^{(1)}(\vk_1,\tau) {\delta}^{(1)}(\vk-\vk_1,\tau) \eql{sorder}
\ee 
\\Discussion (Accuracy?...)

\section{Quadratic Estimator}
Comparison with CMB lensing \cite{123} \\
Compute the correlation function of two short-wavelength modes $\vk_s$ and $\vk_s'$, in the squeezed limit $|\vk_s+\vk_s'|\ll |\vk_s|, \,|\vk_s'|$ up two second-order ($\tau$ is fixed):
\bea 
&& \langle {\delta}(\vec{k}_s){\delta}(\vec{k}_s') \rangle|_{(\vk_s+\vk_s')\neq 0}
\vs 
&=&  \langle {\delta}^{(1)}(\vec{k}_s){\delta}^{(2)}(\vec{k}_s') \rangle+\langle {\delta}^{(2)}(\vec{k}_s){\delta}^{(1)}(\vec{k}_s') \rangle
\eea 
Substituting \ec{sorder} into the first bracket we get:
\bea 
\langle {\delta}^{(1)}(\vec{k}_s){\delta}^{(2)}(\vec{k}_s') \rangle =  \int \frac{d^3\vec{k}}{(2\pi)^3} F_2 (\vec{k},\vec{k}_s'-\vec{k})\vs
\times \langle {\delta}^{(1)}(\vec{k}_s){\delta}^{(1)}(\vec{k}_s'-\vec{k}){\delta}^{(1)}(\vec{k}) \rangle \eql{integral}
\eea 
This correlation function with three Gaussian fields is nonzero under following consideration: if a three-point correlation function consists of two short-wavelength modes and one long-wavelength mode, we can do the following contraction:
\be 
\langle
\contraction{}{{\delta}}{{\delta}(\vk_s'\,}{}
{\delta}(\vk_s) {\delta}(\vk_s'){\delta}(\vk_l) 
\rangle
=\langle {\delta}(\vk_s) {\delta}(\vk_s') \rangle {\delta}(\vk_l) 
\ee 
Since in a real life survey, we can only measure this long-wavelength mode once. Thus it won't have any statistical property and we can safely extract this term out of the bracket.\\
We can do this contraction twice in this integral. One occurs when $|\vk| \ll |\vk_s|,\, |\vk_s'-\vk|$, we can extract ${\delta}^{(1)}(\vk)$ out; the other one occurs when $|\vk_s'-\vk| \ll |\vk|,\,|\vk_s|$.\\
Use \ec{lin} the integral in \ec{integral} can be evaluated as:
\bea 
&&\int \frac{d^3\vec{k}}{(2\pi)^3} F_2 (\vec{k},\vec{k}_s'-\vec{k})\langle {\delta}^{(1)}(\vec{k}_s){\delta}^{(1)}(\vec{k}_s'-\vec{k}){\delta}^{(1)}(\vec{k}) \rangle \vs
&=&\int d^3\vk F_2(\vec{k},\vec{k}_s'-\vec{k}) \delta_{\rm D}(\vk_s+\vk_s'-\vk)P_{\rm 
lin}(k_s){\delta}^{(1)}(\vec{k}) \vs
&+&\int d^3\vk F_2(\vec{k},\vec{k}_s'-\vec{k}) \delta_{\rm D}(\vk_s+\vk)P_{\rm 
lin}(k_s){\delta}^{(1)}(\vk_s'-\vk) \vs
&=&2F_2(-\vk_s,\vk_s+\vk_s')P_{\rm lin}(k_s){\delta}^{(1)}(\vk_s+\vk_s')
\eea 
Finally we have:
\be 
\langle {\delta}(\vec{k}_s){\delta}(\vec{k}_s') \rangle =f(\vec{k}_s,\vec{k}_s'){\delta}^{(1)}(\vec{k}_s+\vk_s') \eql{2pt}
\ee 
with
\bea
f(\vec{k}_s,\vec{k}_s')&=&2F_2(-\vec{k}_s,\vec{k}_s+\vec{k}_s')P_{\rm lin}(k_s)\vs
&+&2F_2(-\vec{k}_s',\vec{k}_s+\vec{k}_s')P_{\rm lin}(k_s')       
\eea 
\ec{2pt} suggests that we can estimate long-wavelength modes with appropriate average over pairs of shart-wavelength modes. General form of the quadratic estimator can be written as:
\begin{eqnarray}
\hat{\delta}^{(1)}(\vec{k}_l)=A(\vec{k}_l)\int \frac{d^3 \vec{k}_s}{(2\pi)^3} g(\vec{k}_s,\vec{k}_s'){\delta}(\vec{k}_s){\delta}(\vec{k}_s')
\end{eqnarray} 
with $g$ being weighting function, $\vk_s'=\vk_l-\vk_s$ and $A$ is defined via $\langle \hat{\delta}^{(1)}(\vec{k}_l) \rangle={\delta}^{(1)}(\vec{k}_l)$:
\begin{eqnarray}
A(\vec{k}_l)=\bigg[\int \frac{d^3 \vec{k}_s}{(2\pi)^3} g(\vec{k}_s,\vec{k}_s')f(\vec{k}_s,\vec{k}_s')  \bigg]^{-1}
\end{eqnarray}
The Gaussian noise is given by:
\be 
\langle \hat{\delta}^{(1)}(\vk_{l})\hat{\delta}^{(1)}(\vk_{l}') \rangle = (2\pi)^3 \delta_{\rm D}[\vk_{l}-\vk_{l}')(P_{\rm lin}(k_{l})+N(\vk_l)]
\ee 
with 
\begin{eqnarray}
&&N(\vec{k}_{l})=2A^2(k_{l})\vs
&&\times\int \frac{d^3 \vec{k}_{s}}{(2\pi)^3} g^2(\vec{k}_{s},\vk_l-\vec{k}_{s})P_{\rm nl}(k_{s})P_{\rm nl}(|\vk_l-\vk_s|)
\end{eqnarray}
where $P_{\rm nl}$ is the nonlinear power spectrum. Minimizing the noise term we can fix the form of $g$ to be:
\begin{eqnarray}
g(\vec{k}_{s},\vec{k}_{s}')=\frac{f(\vec{k}_{s},\vec{k}_{s}')}{2P_{\rm nl}(k_{s})P_{\rm nl}(k_{s}')}
\end{eqnarray} 
Noise term reduces simply to $N(\vk_l)=A(\vk_l)$.
\section{Discussion}


\acknowledgements
We thank XXX for resourceful discussions.  SD and PL were supported by U.S.\ Dept.\ of Energy contract DE-SC0019248.
%\input{main.bbl}
\bibliography{refs}
\end{document}